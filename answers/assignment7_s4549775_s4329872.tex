\documentclass[12pt, a4paper]{article}

\usepackage{amssymb}
\usepackage{multicol}
\usepackage{enumerate}
\usepackage[top=5em, bottom=5em, left=5em, right=5em]{geometry}
\usepackage{listings}
\usepackage{tikz}
\usetikzlibrary{positioning}

\setlength\parskip{1em}
\setlength\parindent{0em}

\title{Assignment 7}

\author{Hendrik Werner s4549775}

\begin{document}
\maketitle

This was done in collaboration with Constantin Blach (s4329872).

\section{} %1
After inserting all values the binary search tree looks as follows:

\begin{tikzpicture}[c/.style={circle, draw, minimum width=2em}]
	\node[c] (A) {9};
	\node[c, below left=8em of A] (B) {5};
	\node[c, below right=8em of A] (C) {12};
	\node[c, below left=of B] (D) {4};
	\node[c, below right=of B] (E) {7};
	\node[c, below left=of C] (F) {10};
	\node[c, below right=of C] (G) {18};
	\node[c, below right=of F] (H) {11};
	\node[c, below right=of E] (I) {8};

	\draw (A) -- (B);
	\draw (A) -- (C);
	\draw (B) -- (D);
	\draw (B) -- (E);
	\draw (C) -- (F);
	\draw (C) -- (G);
	\draw (E) -- (I);
	\draw (F) -- (H);
\end{tikzpicture}

After removing $9$ is looks like this:

\begin{tikzpicture}[c/.style={circle, draw, minimum width=2em}]
	\node[c] (A) {10};
	\node[c, below left=8em of A] (B) {5};
	\node[c, below right=8em of A] (C) {12};
	\node[c, below left=of B] (D) {4};
	\node[c, below right=of B] (E) {7};
	\node[c, below left=of C] (F) {11};
	\node[c, below right=of C] (G) {18};
	\node[c, below right=of E] (I) {8};

	\draw (A) -- (B);
	\draw (A) -- (C);
	\draw (B) -- (D);
	\draw (B) -- (E);
	\draw (C) -- (F);
	\draw (C) -- (G);
	\draw (E) -- (I);
\end{tikzpicture}

After removing $5$ is looks like this:

\begin{tikzpicture}[c/.style={circle, draw, minimum width=2em}]
	\node[c] (A) {10};
	\node[c, below left=8em of A] (B) {7};
	\node[c, below right=8em of A] (C) {12};
	\node[c, below left=of B] (D) {4};
	\node[c, below right=of B] (E) {8};
	\node[c, below left=of C] (F) {11};
	\node[c, below right=of C] (G) {18};

	\draw (A) -- (B);
	\draw (A) -- (C);
	\draw (B) -- (D);
	\draw (B) -- (E);
	\draw (C) -- (F);
	\draw (C) -- (G);
\end{tikzpicture}

\section{} %2
\begin{enumerate}
	\item Every node has $3$ pointers which are all initially $NIL$.
	\label{sec2:ppn}

	\item (\ref{sec2:ppn}) means that in the trivial case of a $1$-node BST there are $3$ $NIL$-pointers.
	\label{sec2:trivial}

	\item Pointers are connected pairwise. $(\forall left_i \neq NIL, \exists p_j \neq NIL) \land (\forall right_i \neq NIL, \exists p_j \neq NIL)$
	\label{sec2:pcp}

	\item (\ref{sec2:ppn}, \ref{sec2:trivial}, and \ref{sec2:pcp}) can be expressed as a recurrence relation:

	$N_1 = 3$\\
	$N_k = N_{k - 1} + 1$

	where $N_k$ is the number of $NIL$-pointers in a $k$-node BST.

	\begin{itemize}
		\item associated homogeneous recurrence relation: $N_k = N_{k - 1}$
		\item characteristic equation: $r - 1 = 0$
		\item $r_1 = 1, m_1 = 1$
	\end{itemize}
	\label{sec2:rr}

	\item (\ref{sec2:rr}) can be expressed as $f(n) = n + 2$.
\end{enumerate}

\section{} %3
\section{} %4
\section{} %5

\end{document}
